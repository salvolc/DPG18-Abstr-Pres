\documentclass{scdpg}
\begin{document}
\scBookLanguage{en}
\begin{scAbstract}
%\scNoUseTeX
\scLanguage{en}
\scTitle{Studies on interference effects in processes with flavor-changing neutral currents and $tq\gamma$ coupling}
\scAuthor{*}{Salvatore La Cagnina}{}{}
\scAuthor{}{Gregor Gessner}{}{}
\scAuthor{}{Johannes Erdmann}{}{}
\scAuthor{}{Kevin Kr\"{o}ninger}{}{}
\scAffiliation{1}{TU Dortmund, Lehrstuhl f\"{u}r Experimentelle Physik IV}
\scBeginText
In the Standard Model, flavor-changing neutral currents (FCNC) at tree level are not present and are highly suppressed by the GIM mechanism at higher orders.
Beyond Standard Model theories, however, can allow FCNCs at tree level.
One possible process containing a FCNC includes a top quark that interacts with an up-type quark and a photon ($tq\gamma$ vertex with $q=u,c$).
It is distinguished between the production mode, in which a single top quark is produced, and the decay mode, in which one of the top quarks of a $t\overline{t}$ system decays through a FCNC interaction.
In next-to-leading order, both modes could interfere.
This interference might lead to changes in the distribution of kinematic variables.
This shown study focuses on the analysis of these interference effects. 
\scEndText
\scConference{W\"{u}rzburg 2018}
\scPart{T}
\scContributionType{Vortrag}
\scTopic{2.08 Top-Quarks: Eigenschaften (Exp.)}
\scEmail{salvatore.lacagnina@tu-dortmund.de}
\scCountry{}
\end{scAbstract}
\end{document}


%, therefore increasing their cross section.
%This vertex can either occur in production mode for single top quarks or as possible decay mode for $t \overline{t}$ processes.
%In the shown studies both processes are generated simultaneously in next-to-leading order.
%This allows both modes to interfere with each other.
%This interference could lead to changes in the distribution of kinematic variables, allowing the interference to come observable.
% The standard model of particle physics describes the fundamental particles and their interactions with each other.
% These interactions are described by the four fundamental forces.
% One of the forces is the weak interaction which allows to change the flavor of the quarks.
% Usually, this process is induced by a charged current by including the exchange of a W Boson.
% However, theories beyond the standard model allow flavor changing processes to occur using a neutral current (FCNC).
% One possible FCNC process includes a top quark changing flavor to an up or charm quark by radiating a photon.
% This vertex can either exist as a production mode for top quarks or as possible decay mode for $t \overline{t}$ processes.
% The focus of this study is use Monte Carlo generators to generate samples with both production and decay mode in next-to-leading order (NLO).
% Generating both processes simultaneously allows for interference effects between the two modes.
% This interference could lead to changes in the distribution of kinematic variables, allowing the interference to be measured.